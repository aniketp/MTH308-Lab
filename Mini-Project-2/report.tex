\documentclass[12pt]{article}

\usepackage[utf8]{inputenc}
\usepackage{fullpage,amsfonts,mathpazo, mathtools, pgfplots}
\usepackage[bottom, hang, flushmargin]{footmisc}
\usepackage{hyperref}

\setlength{\footnotesep}{1.5pc}
\setlength{\parindent}{0pt}

\title{\vspace{-70pt}MTH-308B Mini Project 2}
\author{Aniket Pandey}
\date{Roll No: 160113}
\pagestyle{empty}
\begin{document}

\maketitle

\section*{Problem Statement}
Given a discrete computational grid $X_{grid} = {\{x_j\}}^n_{j=1}$ along with its evaluation data
$F_{grid} = {\{f(x_j)\}}^n_{j=1}$, where $x_j \in [a, b]$, $a < b$. The function $f \in C^{\infty}[a,b]$
and satisfies the property:
    \begin{align}
	f^{(k)}(a) & = f^{(k)}(b) = 0 \qquad \qquad \forall k \in [1, 10] \\
	f^{(11)}(a) & \neq f^{(11)}(b)
    \end{align}
    
Design and implement an efficient polynomial interpolation method to approximate \\
$f_n[a,b] \rightarrow \mathbb{R}$ on evaluation points based on the given data $(X_{grid}, F_{grid})$.

\section*{Proposed Algorithm}
The design implements the method described in the paper, \textit{Finding the Zeros of a Univariate Equation: Proxy Rootfinders, Chebyshev Interpolation, and the Companion Matrix} by \textit{John P. Boyd}.

\subsection*{Implementation Abstract}
When a function $f(x)$ is holomorphic on an interval $x \in [a, b]$, its roots on the interval can be computed
by the following three-step procedure.

\begin{enumerate}
	\item Approximate $f(x)$ on $[a, b]$ by a polynomial $f_N(x)$ using adaptive Chebyshev interpolation.
	\item Form the Chebyshev– Frobenius companion matrix whose elements are trivial functions of the
	Chebyshev coefficients of the interpolant $f_N(x)$.
	\item Compute all the eigenvalues of the companion matrix.
\end{enumerate}

The eigenvalues $\lambda$ which lie on the real interval $\lambda \in [a, b]$ are very accurate approximations to the zeros of $f(x)$ on the target interval.

\newpage
\subsection*{Implementation Details}
Here, we construct Chebyshev interpolation of $f(x)$ to compute a Chebyshev series, including terms up to 
and including $T_N$, on the interval $x \in [a, b]$.

\begin{enumerate}
	\item Create the set of interpolation points: $X_{Grid}$
	\begin{align}
		x_k \equiv \Big(\frac{a+b}{2}\Big) + \frac{b-a}{2}cos\Big(\pi\frac{k}{N}\Big), \qquad  k = 0, 1, 2 ,..., N
	\end{align}
	
	\item Compute the grid point values of $f(x)$, the function to be approximated:
	\begin{equation}
		f_k \equiv f(x_k), \qquad \qquad k = 0, 1, 2 ,..., N
	\end{equation}
	
	\textbf{Note}: Such type of points avoid the periodicity factor that may have caused the error since
	the points are unequally placed and heavily dependent on $X_{grid}$.
	
	\item As mentioned in one of the Lectures (and Lab11), interpolating a polynomial based on
	the points generated by Chebyshev series results in a polynominal which explonentially converges to
	the actual function, as $n$ is increased.
	
	The technique used to generate the interpolation matrix ($nGrid$ X $nGrid$) is:
	\begin{align}
		\Gamma_{jk} = \frac{2}{p_kp_jN} cos \Big(j\pi\frac{k}{N}\Big) \qquad \qquad \qquad
	\end{align}
	
	\item Compute the coefficients through a vector matrix-multiplication
	\begin{align}
		a_j = \sum_{k=0}^{N} \Gamma_{jk}f_k, \qquad \qquad j = 0, 1, 2, ..., N
	\end{align}
	
	\item The function $f(k)$ can then be approximated by inverse interpolation method.
	
	\begin{align}
		f \approx \sum_{j=0}^{N} a_j cos \bigg\{j cos^{-1}\bigg(\frac{2x - (b + a)}{b - a}\bigg)\bigg\}
	\end{align}
 
\end{enumerate}

The vector matrix-multiplication procedure could have been accelerated by \textit{Fast-Fourier Transform},
	but since this vector-matrix multiplication costs $O(2N^2)$ while the eigensolving cost is $O(10N^3)$, the FFT is not worth the effort and computation time.

\subsection*{References}
\begin{enumerate}
	\item Polynomial Interpolation Wiki: \href{https://en.wikipedia.org/wiki/Polynomial_interpolation}{\textit{https://en.wikipedia.org/wiki/Polynomial\_interpolation}}
	\item John P. Boyd,  \textit{Finding the Zeros of a Univariate Equation: Proxy Rootfinders, Chebyshev Interpolation, and the Companion Matrix}
\end{enumerate}

\pagebreak
\end{document}
